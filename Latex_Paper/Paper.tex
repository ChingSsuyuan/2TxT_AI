\documentclass[11pt]{article}
\usepackage[utf8]{inputenc}
\usepackage[T1]{fontenc}
\usepackage{amsmath}
\usepackage{amssymb}
\usepackage{booktabs}
\usepackage{hyperref}
\usepackage{xcolor}
\usepackage{microtype}

\usepackage[margin=1in]{geometry}
\setlength{\parskip}{0.5em}
\setlength{\parindent}{0em}
\makeatletter
\def\@maketitle{
  \newpage
  \null
  \vskip 2em
  \begin{center}
  \let \footnote \thanks
    {\LARGE \@title \par}
    \vskip 1.5em
    {\large \lineskip .5em
      \begin{tabular}[t]{c}
        \@author
      \end{tabular}\par}
    \vskip 1em
    {\large \@date}
  \end{center}
  \par
  \vskip 1.5em}
\makeatother

\begin{document}

\title{Image Caption Generation With CLIP+GPT-2 Model}
\author{Siyuan Jing and Haonan Wu\\ Boston University \\ siyuan16@bu.edu and whn17@bu.edu}
\date{May 3, 2025}
\maketitle

\begin{abstract}
To be finished
\end{abstract}

\section{Introduction}
Image caption generation, the task of recognizing images to generate natural language descriptions, lies at the critical intersection of 
computer vision and natural language processing. 
It plays an important role in a variety of applications, including image retrieval, accessibility for the visually impaired, and automated image content processing.

Recent advances in vision-language models have enabled more accurate and 
fluent caption generation by leveraging large-scale pretraining on aligned image-text pairs 
. These models demonstrate remarkable capabilities in understanding visual content and 
translating it into coherent text descriptions.

Motivated by this progress, we focus our project on replicating a CLIP+GPT-2-based image captioning system, with inspriation from Nukrai et al.\cite{Nukrai2022}. Through this project, we seek to explore and deepen our understanding of 
machine learning, as well as its broader applications in fields such as computer vision and natural language generation.

Our goal is to:
\begin{itemize}
    \item Leverage the pretrained CLIP model to extract semantically rich image embeddings without the need for training a custom vision encoder.
    \item Utilize the generative capabilities of GPT-2 to produce fluent and coherent natural language captions.
    \item Bridge the gap between visual and textual modalities by introducing a projection layer that maps image embeddings into GPT-2's input space.
    \item Enable flexible and data-efficient image captioning, where the visual semantics guide the generation through prefix-based conditioning.
    \item Evaluate the quality of the generated captions using standard metrics such as BLEU and CIDEr, in order to quantitatively assess the model's accuracy and relevance.
\end{itemize}
\pagebreak
\section{Background and Related Work}

    \subsection{Contrastive Language-Image Pretraining(CLIP)}
    CLIP from OpenAI is a visual-language model. Instead of relying on 
    task-specific supervised learning, CLIP is trained on a dataset of 400
    million image-text pairs collected from the internet using a contrastive 
    loss function. CLIP consists of two separate encoders: a visual encoder 
    (ResNet) for images, and a text encoder (Transformer) for captions. 
    Its ability to generate rich, semantically meaningful image 
    embeddings makes CLIP a powerful foundation for our systems and 
    an ideal visual component in our CLIP+GPT-2 image captioning pipeline. 
    For example, according to Mokady et al.\cite{Mokady2021}, it mentioned that the visual encoding capability of 
    CLIP can be used to embed and project the generated images into the input 
    space of GPT-2 to generate prefixes, which helps the final caption generation of GPT-2. Inspired by this article, we decided to study the CLIP architecture and implement related deployments.

    \subsection{Generative Pre-trained Transformer 2 (GPT-2)}
    GPT-2 is a large-scale language model based on the Transformer 
    decoder architecture proposed by Radford et al.\cite{Radford}. 
    According to the paper, Transformer completely replaces the traditional RNN or 
    CNN structure with self-attention, which is more efficient and accurate when processing 
    long sequence dependencies. Therefore, we consider implementing the transformer 
    structure as our decoder of the whole pipeline. Unlike the traditional 
    Transformer, which contains both encoder and decoder components, 
    GPT-2 uses only a decoder. This design enables the model to predict the next token based solely on previously generated tokens, 
    making the generated text semantically relevant and well suited for text generation tasks such as image captioning.
    
    \subsection{Multi-Head Attention Mechanism}
    Inspired by the \textit{``Attention Is All You Need''} paper 
    \cite{vaswani2017attention}, we employ a multi-head attention 
    mechanism as a cross-modal connector between the CLIP image embedding
    and the GPT-2 language model. Instead of using a linear layer to directly 
    project the CLIP features into the GPT-2 embedding space, we employ an 
    attention mechanism that enables the model to selectively focus on 
    different aspects of the visual features when generating each token 
    in the caption, making the generated tokens more consistent with the 
    image information. Multi-head Attention network allows the model to 
    learn how different regions in the image affect language production.

We compare these advanced approaches of image caption generation to understand its 
advantages and limitations, and implement these methods in the process of model training to get our own model.
\pagebreak
\section{Methodology}

\subsection{Implementation Details}
\textbf{Detail of Implementation and pipeline:} This is the pipeline of 
our entire training process, which mainly includes multiple steps: 
image input, feature extraction and encoding, feature processing, 
feature decoding, model training and error evaluation. We will also 
introduce the optimization we used.


\textbf{Augmentation Pipeline:} SimCLR relies heavily on augmentations, which include:
\begin{itemize}
  \item Random crop and resize
  \item Color jittering
  \item Gaussian blur
  \item Horizontal flipping
\end{itemize}
\begin{equation}
    L = -\sum_{i \in I} \log \frac{\exp(\text{sim}(z_i, z_j)/\tau)}{\sum_{k \neq i}\exp(\text{sim}(z_i, z_k)/\tau)}
    \end{equation}
    where $\text{sim}(z_i, z_j)$ is the cosine similarity between embeddings, and $\tau$ is the temperature hyperparameter.
    
\textbf{Training Details:} We train our model using these hyperparameter after cross-validation:
\begin{itemize}
  \item Prefix length: 40
  \item Transformer layers: 8
  \item Optimizer: Adam with a learning rate of $5 \times 10^{-5}$
  \item Number of epochs: 20
  \item Noise rate: 10\%
\end{itemize}

\subsection{Evaluation Protocol}
We evaluate the quality of learned representations by training a simple linear classifier on top of the
frozen embeddings.

\section{Experiments and Results}

\subsection{Dataset and Preprocessing}
We conduct experiments on:
\begin{itemize}
  \item \textbf{CIFAR-10:} A 10-class dataset with 60,000 images.
  \item \textbf{STL-10:} A larger dataset often used for unsupervised learning benchmarks.
\end{itemize}

\subsection{Baseline Comparisons}
We compare SimCLR embeddings with:
\begin{itemize}
  \item PCA-based dimensionality reduction ($d = 128$)
  \item Autoencoders trained on the same dataset
  \item Supervised ResNet-18 trained on CIFAR-10
\end{itemize}

\begin{table}[h]
\centering
\begin{tabular}{lcc}
\toprule
Method & CIFAR-10 Accuracy (\%) & STL-10 Accuracy (\%) \\
\midrule
Supervised ResNet-18 & 92.5 & 85.4 \\
PCA + kNN & 45.6 & 38.2 \\
Autoencoder + kNN & 55.3 & 49.6 \\
SimCLR (Ours) & 80.2 & 76.4 \\
\bottomrule
\end{tabular}
\caption{Comparison of representation learning methods. SimCLR significantly outperforms classical techniques.}
\end{table}

\subsection{Ablation Studies}
\textbf{Effect of Temperature $\tau$:} We analyze how different values of $\tau$ in the contrastive loss impact performance.

\section{Discussion}

\subsection{Key Findings}
\begin{itemize}
  \item SimCLR significantly outperforms PCA and autoencoders in feature learning.
  \item The choice of augmentations greatly affects performance.
  \item Higher temperature values in contrastive loss lead to better separation of features.
\end{itemize}

\subsection{Future Work}
\begin{itemize}
  \item Extend to other self-supervised methods (e.g., BYOL, MoCo).
  \item Apply to domain adaptation tasks.
  \item Explore contrastive learning for text or multimodal applications.
\end{itemize}

\section{Conclusion}
Our empirical study demonstrates the effectiveness of contrastive learning via SimCLR for representation
learning. By systematically evaluating augmentation pipelines, batch sizes, and loss functions, we provide
insights into optimizing contrastive learning for different datasets.
\pagebreak
\begin{thebibliography}{99}
    \bibitem{Nukrai2022}
    Nukrai, D., Mokady, R., \& Globerson, A. (2022). Text-Only Training for Image Captioning using Noise-Injected CLIP. https://doi.org/10.48448/n7sq-p557
    
    \bibitem{Mokady2021}
    Mokady, R., Hertz, A., \& Bermano, A. H. (2021). ClipCap: CLIP Prefix for Image Captioning. http://arxiv.org/abs/2111.09734
    
    \bibitem{Radford}
    Radford, A., Wu, J., Child, R., Luan, D., Amodei, D., \& Sutskever, I. (2019). ``Language Models are Unsupervised Multitask Learners.'' OpenAI Technical Report.

    \bibitem{vaswani2017attention}
    Vaswani, A., Shazeer, N., Parmar, N., Uszkoreit, J., Jones, L., Gomez, A. N., Kaiser, Ł., \& Polosukhin, I. (2017). ``Attention is all you need.'' 
    Advances in Neural Information Processing Systems, 30.

    \bibitem{mokady2021clipcap}
    Mokady, R., Hertz, A., \& Bermano, A. H. (2021). ``ClipCap: CLIP Prefix for Image Captioning.'' arXiv preprint arXiv:2111.09734.
    
    \bibitem{papineni2002bleu}
    Papineni, K., Roukos, S., Ward, T., \& Zhu, W. J. (2002). ``BLEU: a method for automatic evaluation of machine translation.'' ACL 2002.
    
    \bibitem{vedantam2015cider}
    Vedantam, R., Lawrence Zitnick, C., \& Parikh, D. (2015). ``CIDEr: Consensus-based image description evaluation.'' CVPR 2015.
    
    \bibitem{chen2020simple}
    Ting Chen, Simon Kornblith, Mohammad Norouzi, Geoffrey Hinton. ``A Simple Framework for Contrastive Learning of Visual Representations.'' ICML 2020.
    
    \bibitem{hinton2006reducing}
    Geoffrey Hinton, Ruslan Salakhutdinov. ``Reducing the Dimensionality of Data with Neural Networks.'' Science, 2006.
    \end{thebibliography}

\end{document}